\documentclass[10pt]{beamer}

\usetheme{Frankfurt}
\useoutertheme{default}
\setbeamertemplate{footline}[page number]
\setbeamertemplate{navigation symbols}{} 

\usepackage{cmap}					% поиск в PDF
\usepackage[T2A]{fontenc}			% кодировка
\usepackage[utf8]{inputenc}			% кодировка исходного текста
\usepackage[english,russian]{babel}	% локализация и переносы
\usepackage{indentfirst}
\frenchspacing

\newcommand{\delayV}[1]{\overset{\leftarrow}{\textbf{x}}_{#1}}

\theoremstyle{definition}
\newtheorem*{Def}{Определение}

\title{Классификация временных рядов в пространстве модели c подходом NeuralODE}
\author{Сёмкин Кирилл}

\institute[MIPT]{Московский физико-технический институт \\ Кафедра интеллектуальных систем}

\date[2025]{\textit{Научный руководитель}: д.ф.-м.н. Стрижов Вадим Викторович \\ 2025}

\begin{document}
	
	\begin{frame}[c]
		\titlepage
	\end{frame}
	
	\begin{frame}{Проблематика работы}
		
		\begin{alertblock}{Проблема}
			Необходим метод классификации временных рядов, порождаемых скрытыми динамическими системами. Классификация без учёта порождения данных может быть неустойчивой и некорректной.
		\end{alertblock}
		
		\begin{block}{Цель}
			Ввести вероятностную постановку порождения временных рядов в связке с моделью \emph{ОДУ}. Решить проблему ненаблюдаемости порождающих динамических систем. Сформулировать формальную задачу классификации и предложить способы решения.
		\end{block}
		
		\begin{exampleblock}{Решение}
			% Использовать \emph{теорему Такенса} для восстановления зашумлённых фазовых траекторий,
			Использовать \emph{NeuralODE} для аппроксимации динамических систем. Параметры системы могут быть фиксированными или порождаться \emph{априорным} распределением. Классификацию осуществлять с помощью \emph{байесовского тестирования гипотез} или строить классификатор в пространстве параметров дин. системы.
		\end{exampleblock}
		
	\end{frame}	
	
	\begin{frame}{Постановка задачи}
		
		Задана обучающая и тестовая выборка временных рядов для каждого класса. Количество классов $K$.
		
		Пусть для каждого класса существует динамическая система $\textbf{f}_i$, порождающая траектории $\textbf{z}(t)$ что
		
		\begin{equation*}
			\begin{cases}
				\dfrac{d \textbf{z}}{dt} (t) = \textbf{f}_i \big( \textbf{z}(t) \big), \\
				\textbf{z}(0) = 0.
			\end{cases}
		\end{equation*}
		
		Пусть существует \emph{функция наблюдений} $\phi$: $\phi\big( \textbf{z}(t) \big) = x(t)$.
		
		\begin{figure}
			\centering
			\includegraphics[width=0.7\textwidth,keepaspectratio]{img/phi_func.png}
		\end{figure}
		
	\end{frame}	
	
	\begin{frame}{Восстановление и параметризация дин. системы}
		
		Наложив некоторые условия регулярности на $\textbf{f}_i$ и $\phi$, с помощью теоремы Такенса можем получить \emph{вложение} исходных дин. систем в $\mathbb{R}^m$. Фазовыми траекториями будут \emph{вектора задержки} $\delayV{t}$
		
		\begin{equation*}
			\textbf{z}(t) = \delayV{t} := \begin{pmatrix}
				x(t - L + 1) \\
				\vdots \\
				x(t - 1) \\
				x(t)
			\end{pmatrix} \in \mathbb{R}^m.
		\end{equation*}
		
		Предположим, что векторные поля $\textbf{f}_i$ лежат в известном параметрическом классе, т.е. $\textbf{f}_i = \textbf{f}_{\Theta_i}$. Наконец наложим на тректории независимый шум с нулевым средним и ограниченной дисперсией
		
		\begin{gather*}
			\textbf{z}(t) \to \textbf{z}(t) + \boldsymbol{\epsilon}, \\
			\text{s.t. } \mathbb{E}[\boldsymbol{\epsilon}] = 0, \, \mathbb{D}[\boldsymbol{\epsilon}] < +\infty.
		\end{gather*}
		
	\end{frame}	
	
	\begin{frame}{Полная модель порождения данных}
		
		Предлагаются три типа связи параметров $\Theta_i$ с дин. системами $\textbf{f}_i$ в рамках класса:
		
		\begin{enumerate}
			\item Дин. система класса имеет фиксированные параметры $\Theta_i = $ const. 
			
			\item Класс имеет \emph{априорное} распределение на параметры $\Theta_i \sim p_i(\Theta)$ (генеративная модель).
		\end{enumerate}
		
		Добавив априорное распределение на классы $ C \sim \text{Cat}(C) $, мы полностью определим вероятностную модель задачи.
		
		\begin{enumerate}
			\setcounter{enumi}{2}
			\item Каждый параметр $\Theta$ задаёт распределение на класс $C \sim p(C | \Theta)$ (дискриминативная модель).
		\end{enumerate}
		
		\begin{figure}
			\centering
			\includegraphics[width=0.49\textwidth,keepaspectratio]{img/bayes_graph_gen.jpg}
			\includegraphics[width=0.49\textwidth,keepaspectratio]{img/bayes_graph_discr.png}
			\caption{Графические модели для 2 и 3 типов связи}
		\end{figure}
		
	\end{frame}

	\begin{frame}{Алгоритмы классификации временных рядов}
		
		Для каждого типа связи:
		
		\begin{enumerate}
			\item На обучающей выборке получить ML оценки параметров класса $\hat{\Theta}_i$. Для тестовой траектории воспользоваться \emph{байесовским решающим правилом}:
			
			\begin{equation*}
				C_{\text{test}} = \underset{C_i}{\arg \max} \, p(C_i) p\big( \textbf{z}_{\text{test}}(t) | \hat{\Theta}_i \big)
			\end{equation*} 	
			
			\item Байесовский вывод
			
			\begin{equation*}
				p \big(C = C_i | \textbf{z}_{\text{test}}(t), \textbf{z}_{\text{train}}(t) \big) = \int p \big( C = C_i | \Theta, \textbf{z}_{\text{test}}(t) \big) p_i \big( \Theta | \textbf{z}_{\text{train}}(t) \big) d\Theta
			\end{equation*}
			
			\item По каждой обучающей траектории получить ML оценку порождающей дин.системы $\hat{\Theta}$. Далее, на полученных оценках обучить классификатор в пространстве параметров $p(C | \Theta)$. Для тестовой траектории снова получаем оценку $\hat{\Theta}_{\text{test}}$, пользуемся классификатором:
			
			\begin{equation*}
				C_{\text{test}} = \underset{C_i}{\arg \max} \, p(C_i | \hat{\Theta}_{\text{test}})
			\end{equation*}
		\end{enumerate}
		
	\end{frame}	
	
	\begin{frame}{Постановка эксперимента}
		
		\begin{block}{Цель эксперимента}
			Восстановить фазовые траектории и скрытые дин. системы по обучающей выборке, сравнить качество классификации тремя предложенными методами + c моделями RNN, CNN. Оценить применимость предложенных методов.
		\end{block}
		
		\begin{exampleblock}{Данные}
			Акселерометрия движений человека для разных типов активностей (50 Гц, 6 классов), датасет \href{https://github.com/mmalekzadeh/motion-sense}{MotionSense}.
		\end{exampleblock}
		
		\begin{figure}[h]
			\centering
			\includegraphics[width=0.5\textwidth]{img/motionsense}
			\caption{Примеры временных рядов в MotionSense}
		\end{figure}
		
	\end{frame}	
	
	\begin{frame}{Фазовые траектории}
		
		Временные ряды активности "upstairs" и восстановленные фазовые траектории.
		
		\begin{figure}
			\centering
			\includegraphics[width=0.4\textwidth]{img/series_ups.png}
			\includegraphics[width=0.59\textwidth]{img/phase_ups.png}
		\end{figure}
		
	\end{frame}	
	
	\begin{frame}{Классификация в пространстве параметров}
		
		Классы активности "jog" и "stand". Для каждой траектории обучалась линейная модель. В пространстве параметров построен kNN-классификатор. Приведены метрики качества для каждого класса (тестовая выборка) и визуализация t-SNE обученных моделей.
		
		\begin{table}[]
			%\caption{Качество классификации в пространстве параметров}
			\begin{tabular}{|c|c|}
				\hline
				\textbf{Label} & \textbf{Accuracy} \\ \hline
				jog            & 0.4               \\ \hline
				std            & 0.7               \\ \hline
			\end{tabular}

		\end{table}
		
		 \begin{figure}
		 	\centering
		 	\includegraphics[width=0.5\textwidth]{img/tsne.png}
		 	\caption{t-SNE на параметрах обученных моделей. Perplexity $= 15$.}
		 \end{figure}
		
	\end{frame}
	
	\begin{frame}{Байесовское решающее правило}
		
		Классы активности "jog" и "upstairs" с равномерным априорным расределением. Траектории поделены на train/test. NN-модели обучены на train. Тестовая траектория классифицируется по наибольшему правдоподобию у обученной модели. Приведены метрики качества для каждого класса (тестовая выборка), примеры аппроксимации реальной траектории обученными моделями.
		
		\begin{table}[]
			\begin{tabular}{|c|c|}
				\hline
				\textbf{Label} & \textbf{Accuracy} \\ \hline
				jog            & 0.68              \\ \hline
				ups            & 0.32              \\ \hline
			\end{tabular}
		\end{table}
		
		\begin{figure}
			\centering
			\includegraphics[width=0.49\textwidth, keepaspectratio]{img/ups_traj_approx.png}
			\includegraphics[width=0.49\textwidth, keepaspectratio]{img/jog_traj_approx.png}
		\end{figure}
		
	\end{frame}
	
	\begin{frame}{Выносится на защиту}
		
		\begin{enumerate}
			\item Поставлена задача классификации временных рядов, порождённых скрытыми дин. системами
			\item Предложены три вероятностные модели порождения фазовых траекторий
			\item Для каждой модели предложен алгортим классификации новых траекторий
			\item Поставлены вычислительные эксперименты по восстановлению параметров дин. систем и классификации 
		\end{enumerate}
		
		В ближайшее время будут проводиться эксперименты на данных гироскопа, где должно получиться более высокое качество классификации. Будет оформлена теор. оценка на точность классификации в связи с численной ошибкой  ode-солвера. Работа оформляется для будущей публикации.
		
	\end{frame}	
	
\end{document}